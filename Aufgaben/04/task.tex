\section{Aufgabe 4}
    \subsection{a)}
        $$f(x) = x^2+(-3-2i)x+5+i$$
        $$0 = x^2+(-3-2i)x+5+i$$
        Wir substituieren $x=a+bi$:
        $$\Rightarrow (a+bi)^2+(-3-2i)(a+bi)+5+i=0$$
        $$=a^2+2abi-b^2-3a-3bi-2ai+2b+5+i=0$$
        $$(a^2-b^2-3a+2b+5)+i(-2a-3b+2ab+1)=0+0i$$
        Da $a+bi=c+di$ nur dann gilt, wenn: $a=c, b=d$. Somit bekommen wir das System: \\
        $I:a^2-b^2-3a+2b+5=0$ \\
        $II:-2a-3b+2ab+1=0$ \\
        $\Rightarrow a_1=1,b_1=-1$ und $a_2=2,b_2=3$ \\
        Durch Rücksubstitution bekommen wir: \\
        $x_1=1-i,x_2=2+3i$

    \subsection{b)}
        $(1-i)^{20}$: Sei $z=1-i \Rightarrow z=re^{i\varphi}$ 
        $$|z|=\sqrt{1^2-i^2}=\sqrt{1-(-1)}=\sqrt{2}\Rightarrow r=\sqrt{2}$$
        $$\varphi=arg(1-i)=-\frac{\pi}{4}$$
        $$\Rightarrow z = \sqrt2 exp(i\frac{-\pi}{4})$$
        $$\Rightarrow z^{20}=\sqrt2^{20}e^{-i\frac{20\pi}{5}}=1024e^{-i5\pi}=1024(cos(-5\pi)+isin(-5\pi)=1024\cdot(-1)=-1024$$
        